%!TEX root = ../report.tex

\begin{document}
    \chapter{State of the Art}

    \section{Perception}
    
    For a robot to work in uncertain environments and generalize itself to a wide variety of tasks, it must be able to sense and understand its environment. The task of processing the sensory data obtained from the various sensors into usable information is called as 'Perception'.
    
    Reliable perception is very important in robotics as most of the further operations in the pipeline will be dependent on the success and quality of the perception task. Modern robots have a wide array of perception sensors which can be used to understand its environment. The sensors help the robot to have “situational awareness” so that it can perform autonomously in a dynamically changing real world environment.
    
    This report will mainly focus on the perception algorithms that can be helpful for solving the task of the manipulating the handles.
    
    Some of the common sensors hardware found in robots include 2D camera, RGB-D camera and Lidar. Different types of sensors have their own advantages and disadvantages with respect to the task they can perform.
    
    General perception tasks usually deals with object detection and recognition. In robotics the perception also plays an important role in localization of the object with respect to the robot and understand the properties (axis of the object, best grasp position) of the object so that it can be appropriately manipulated.
    
    \subsection{Relative and Global Sensors}\cite{2011}
    Sensors can be broadly categorized into Relative and Global sensors. The output of a global sensors are the direct reading of the state, for example the GPS (Global positioning system). Most of the perception sensors found in robotics like Lidars and RGB sensors on the other hand are relative sensors as these itself don't give the value of the required state.  
    
    \section{Limitations of previous work}
    
    \section{Manipulation}

    
\end{document}
