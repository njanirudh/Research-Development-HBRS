%!TEX root = ../report.tex

\begin{document}
    \chapter{Introduction}
    
    “In robotics, the easy problems are hard and the hard problems are easy”  - S. Pinker. 
    (The Language Instinct. New York: Harper Perennial Modern Classics, 1994)

    \section{Motivation}
    Manipulation of different types of Articulated objects are an important skill for robots that assist humans. Handles are a type of articulated objects that are found in most kinds of real-world  domestic and industrial environments.
    
    Many of the common handles found in domestic environments are used to manipulate a connected object, which are usually doors or cabinets in domestic environments. The force that is applied to the handles are dependent on the kinetic model of the object like sliding door, prismatic cabinets, revolute doors and vertically opening cabinets. Ideally the robot must be able to manipulate the handle based on the best possible model that defines the underlying constrained object.
    
    Despite the task of manipulation of handles have been extensively studied, it still remains an open problem with not a single method that can work under all conditions. The vast variation in the type, size and orientation of the joint makes it infeasible to provide the models of all the possible handles. Apart from this, there are also cases where it might not be possible to estimate the best possible kinematic model by simply observing the handle.
    
    Unlike robots, humans can take advantage of the prior experiences to work in new and uncertain environments. For the robot to work autonomously in different conditions the robot must be able to learn to solve the task at hand by exploring the different possibilities and exploiting its prior knowledge.

    \section{Challenges and Difficulties}
    \subsection{...}
    



    \section{Problem Statement}
    This report tries to summarize the methods found in literature for solving the problem of manipulation of handles while finding the characteristics along two subtasks : the perception subtask and manipulation subtask. Based on the collected data about the different methods, this report hopes to serve as a starting point for solving the task of manipulation of handles in domestic environment. 
    
    
\end{document}
